%%%%%%%%%%%%%%%%%%%%%%%%%%%%%%%%%%%%%%%%%
% University Assignment Title Page 
% LaTeX Template
% Version 1.0 (27/12/12)
%
% This template has been downloaded from:
% http://www.LaTeXTemplates.com
%
% Original author:
% WikiBooks (http://en.wikibooks.org/wiki/LaTeX/Title_Creation)
%
% License:
% CC BY-NC-SA 3.0 (http://creativecommons.org/licenses/by-nc-sa/3.0/)
% 
% Instructions for using this template:
% This title page is capable of being compiled as is. This is not useful for 
% including it in another document. To do this, you have two options: 
%
% 1) Copy/paste everything between \begin{document} and \end{document} 
% starting at \begin{titlepage} and paste this into another LaTeX file where you 
% want your title page.
% OR
% 2) Remove everything outside the \begin{titlepage} and \end{titlepage} and 
% move this file to the same directory as the LaTeX file you wish to add it to. 
% Then add \input{./title_page_1.tex} to your LaTeX file where you want your
% title page.
%
%%%%%%%%%%%%%%%%%%%%%%%%%%%%%%%%%%%%%%%%%
%\title{Title page with logo}
%----------------------------------------------------------------------------------------
%	PACKAGES AND OTHER DOCUMENT CONFIGURATIONS
%----------------------------------------------------------------------------------------

\documentclass[12pt]{article}
\usepackage[english]{babel}
\usepackage[utf8]{inputenc}
\usepackage{amsmath}
\usepackage{graphicx}
\usepackage[colorinlistoftodos]{todonotes}
\usepackage{titlesec}
\usepackage{float}
%to include references in toc
\usepackage[nottoc,numbib]{tocbibind}
%bibliography package
\usepackage[backend=bibtex,style=nature]{biblatex}
%----------------------------------------------------------------------------------------
%	BIBLIOGRAPHY
%----------------------------------------------------------------------------------------
% \usepackage[
% backend=biber,
% style=nature,
% ]{biblatex}
% \addbibresource{ThesisProposal.bib}
% \bibliographystyle{nature}
\bibliography{ThesisProposal.bib}

\begin{document}

\begin{titlepage}

\newcommand{\HRule}{\rule{\linewidth}{0.5mm}} % Defines a new command for the horizontal lines, change thickness here

\center % Center everything on the page



 
%----------------------------------------------------------------------------------------
%	HEADING SECTIONS
%----------------------------------------------------------------------------------------
\includegraphics[width=0.3\columnwidth]{University_of_Los_Andes_logo.png}\\[1cm] %
\textsc{\Large Universidad de los Andes}\\[1.5cm] % Name of your university/college
\textsc{\large Undergraduate Thesis Proposal}\\[0.5cm] % Major heading such as course name
% \textsc{\large Minor Heading}\\[0.5cm] % Minor heading such as course title

%----------------------------------------------------------------------------------------
%	TITLE SECTION
%----------------------------------------------------------------------------------------

\HRule \\[0.4cm]
{ \Large \bfseries Transient object classification using machine learning and deep learning techniques on real data}\\[0.4cm] % Title of your document
\HRule \\[1.5cm]
 
%----------------------------------------------------------------------------------------
%	AUTHOR SECTION
%----------------------------------------------------------------------------------------

\begin{minipage}{0.4\textwidth}
\begin{flushleft} \large
\emph{Author:}\\
Mauricio Neira % Your name
\end{flushleft}
\end{minipage}
~
\begin{minipage}{0.5\textwidth}
\begin{flushright} \large
\emph{Supervisor:} \\
Marcela Hernández Hoyos, Ph.D. % Supervisor's Name
\end{flushright}
\end{minipage}\\[2cm]

% If you don't want a supervisor, uncomment the two lines below and remove the section above
%\Large \emph{Author:}\\
%John \textsc{Smith}\\[3cm] % Your name

%----------------------------------------------------------------------------------------
%	DATE SECTION
%----------------------------------------------------------------------------------------

{\large \today}\\[2cm] % Date, change the \today to a set date if you want to be precise

%----------------------------------------------------------------------------------------
%	LOGO SECTION
%----------------------------------------------------------------------------------------

\tableofcontents
 
%----------------------------------------------------------------------------------------

\vfill % Fill the rest of the page with whitespace

\end{titlepage}


\section{Introduction}



\subsection{Transient objects}


Two groups [1,2] have presented strong evidence that the expansion of the Universe is speeding up, rather than
slowing down. It comes in the form of distance measurements to some fifty supernovae of type Ia (SNe Ia),


some stufff \cite{SN_darkEnergy}
\subsubsection{Supernovae}
\subsection{LSST}
\section{Problem description}
\section{Project Background}

\subsection{Diego's thesis - 2018-10}

In the first semester of 2018, Diego Alejandro G\'omez Mosquera worked on ``Astronomical transient event recognition with machine learning'' using random forests on a feature space calculated from light curves \cite{diegoThesis}.The author focused on creating a feature space robust enough to distinguish the different transient classes. This was achieved primarily through geometric parameters that were extracted from the light curves. The features used (as found on the author's thesis) were: 
\begin{itemize}
  \item skew: Skewness.
  \item kurtosis: Kurtosis.
  \item small kurtosis: Small sample kurtosis.
  \item std: Standard deviation.
  \item beyond1std: Percentage of magnitudes beyond one standard deviation from the weighted mean. Each weights is calculated as the inverse of the corresponding photometric error.
  \item stetson j: The Welch-Stetson J variability index [39]. A robust standard deviation.
  \item stetson k: The Welch-Stetson K variability index [39]. A robust kurtosis measure.
  \item max slope: Maximum absolute slope (delta magnitude over deltatime) between two consecutive observations.
  \item amplitude: Difference between maximum and minimum magnitudes.
  \item median absolute deviation:from the median magnitude.
  \item median buffer range percentage: Percentage of points within 10\% of the median magnitude.
  \item pair slope trend: Percentage of all pairs of consecutive magnitude measurements that have positive slope.
  \item percent amplitude: Largest percentage difference between the absolute maximum magnitude and the median.
  \item percent difference flux percentile: Ratio of F 5,95 and the median flux.
  \item flux percentile ratio mid20: Ratio F 40,60 /F 5,95
  \item flux percentile ratio mid35: Ratio F 32.5,67.5 /F 5,95
  \item flux percentile ratio mid50: Ratio F 25,75 /F 5,95
  \item flux percentile ratio mid65: Ratio F 17.5,82.5 /F 5,95
  \item flux percentile ratio mid80: Ratio F 10,90 /F 5,95
  \item poly1 a: Coefficient of the linear term in monomial curve fitting.
  \item poly2 a: Coefficient of the cuadratic term in cuadratic curve fitting.
  \item poly2 b: Coefficient of the linear term in cuadratic curve fitting.
  \item poly3 a: Coefficient of the cubic term in cubic curve fitting.
  \item poly3 b: Coefficient of the cuadratic term in cubic curve fitting.
  \item poly3 c: Coefficient of the linear term in cubic curve fitting.
  \item poly4 a: Coefficient of the quartic term in quartic curve fitting.
  \item poly4 b: Coefficient of the cubic term in quartic curve fitting.
  \item poly4 c: Coefficient of the cuadratic term in quartic curve fitting.
  \item poly4 d: Coefficient of the linear term in quartic curve fitting.
\end{itemize}

The results from the authors thesis will not be presented here as a bug overrating the classification was found as will be discussed in the following section.

\subsection{research internship?? No se como ponerle a esto 2018-2}

\subsubsection{Improvement on Diego's work}\label{improvementDiego}
Diego's work was continued and improved during this semester with the mentorship of Marcela Hern\'andez, Jaime Forero and Pablo Arbelaez. 

\subsubsection{PLAsTiCC - Kaggle competition}


\section{General objective}
\section{Specific objectives}

\subsection{Improvement of the light curve feature space}

\subsubsection{Feature pruning}

After seeing the results in \ref{improvementDiego}, \textbf{REVISAR ESTO}, it was clear that the higher order coefficients in the polynomial fits did not contribute significantly to the correct classification in the feature space. In fact, these features could be harming the classification process. Thus, coefficients resulting from $3^{rd}$ and $4^{th}$ degree polynomial fitting will be removed and the random forest algorithm will be rerun. The classification metrics should improve but experimentation is needed to confirm the hypothesis.  

\subsubsection{Addition of supernovae specific metrics}

To improve the classification of supernovae, metrics that target their specific light curve behavior are needed. In particular, there are functions that are known to approximate the light curve produced by a supernova. These functions are presented below: 

\paragraph{SALT2} is ``an empirical model of Type Ia supernovae spectro-photometric evolution with time''\cite{salt2}. This model should be better adjusted for type 1A supernovas than the rest of objects. The mathematical model of the function is the following\cite{salt2}:

\[
  F(S,N,p,\lambda) = x_0 \times [M_0(p,\lambda)+x_1M_1(p,\lambda)+...]\times exp[cCL(\lambda)]
\]

\paragraph{Skewed Gaussian} fits of the form:

\[f^k(t) = A^k \frac{exp-(t-t_0^k/\tau^k_{fall})}{1+exp-(t-t^k_0)\tau^k_{rise}}\]

have been shown to resemble well a generalized supernova curve \cite{sGaussian}. It should also approximate the shape of supernova light curves better than those that belong to other classes.\\

When these functions are fit to supernovae data, the resulting $\chi^2$ should be considerably lower than the $\chi^2$ calculated from non-supernovae fits. Consequently, the addition of the two $\chi^2$  values from each of the functions should improve the binary classification of supernovae but as stated above, experimentation is needed for verification.

\subsection{Deep learning on light curves}

For the time being, the classification pipeline has had 3 steps:

\begin{enumerate}
  \item Clean and filter light curves
  \item Calculate features from clean light curves
  \item Classify the objects on the feature space
\end{enumerate}

When the feature space is calculated, a large quantity of information is lost. The features might be good descriptors of the objects but they will never be able to encapsulate the point-by-point data that the light curves have. Additionally, traditional machine learning methods, like the random forest classifier previously implemented, are unable to handle varying length input. To take advantage of all the information present in the light curves, a new approach is needed.\\

Recurrent neural networks (RNN) have been shown to be able to classify sequential data to a good extent. Speech recognition has been one of the fields with most progress\cite{RNN}. A RNN like the long short term memory (LSTM) RNN or the gated recurrent unit (GRU) RNN are state of the art RNN's that seem promising for this purpose. Thus, several implementations of these networks will be carried out along with their hyperparameter tuning. 

\subsection{Deep learning on images}

\subsubsection{CNNs}

So far, all the classification has been done on the objects' photometric light curves. The light curves, however, are not the raw data. These were calculated from the images that were taken by the telescopes. \textbf{AGAIN, LINK HERE}. Ideally, to minimize data loss, the classification process should be done on the raw data i.e. the images.\\ 

The state of the art algorithms used for classifying images are known as convolutional neural networks (CNNs) \cite{CNN} and have had wide success on large variety of problems.\\

The first step to correctly classify the images will be to implement a baseline CNN algorithm. Once its classification  metrics are established, a thorough search through the hyperparameter space will be carried out to maximize the classification metrics.

\subsubsection{CNNs and RNNs}

Implementing a succesful CNN is only half of the problem. Each object has multiple images taken at different instances in time. To fully exploit the available information, multiple images need to be used as input for each object.\\

Since the amount of images per object is not fixed, a RNN will need to be used. Thus, the CNN will extract descriptors from the images and then those descriptors will be fed in chronological order along with the date of when the image was taken into the RNN for classification. The architecture is depicted below:

\begin{figure}[H]
  \centering
  \includegraphics[width=1.05\textwidth]{CNNRNNDiagram.png}
  \caption{Architecture for classifying astronomical objects from multiple images.}
\end{figure}

\section{Activities and schedule}
\section{Expected results}


\newpage
\printbibliography

\end{document}